%%%%%
%%
%% This file creates the Item, ItemPacket, ItemFold, ItemEnvelope, and
%% ItemLabel datatypes, and creates macros for each.  These are for
%% various types of in-game items.
%%
%%%%%


%%%%%
%% Item macros are for normal item cards.
\DECLARESUBTYPE{Item}{TransElement}
\PRESETS{Item}{
  \FD\MYtext	{} %% longer text of item
  \FD\MYmark	{} %% possible contents of shaded ``mark'' on card
  \FD\MYbulky	{0} %% potential bulkiness
  \FD\MYcapacity{N/A} %% potential capacity
  \sd\MYlistmap	{\item\MYname\ifx\MYnumber\empty\else\ (\MYnumber)\fi}
  }


%%%%%
%% \prop
%% \unstash
%% \bulky{<number>}
%% \contain{<number>}
%%
%% \prop inside an Item macro labels the card as a prop.  \unstash
%% labels the card as unstashable.  \bulky{n} labels the card as
%% n-hands bulky.  \contain{n} labels the card with n-hands capacity.
\def\prop{%
  \append\MYmark{ ~PROP~ }}
\def\unstash{%
  \append\MYmark{ ~UNSTASHABLE~ }}
\def\bulky#1{%
  \s\MYbulky{#1}%
  \append\MYmark{\mbox{ ~\MYbulky-Hand~Bulky~ }}}
\def\contain#1{%
  \s\MYcapacity{#1}%
  \append\MYmark{\mbox{ ~\MYcapacity-Hand~Capacity~ }}}


%%%%%
%% ItemPacket macros are for item cards with an attached packet.
%% They are a subtype of Item.
\DECLARESUBTYPE{ItemPacket}{Item}
\PRESETS{ItemPacket}{
  \F\MYcontents
  }


%%%%%
%% ItemFold macros are for items represented by just a folded packet.
%% They are a subtype of ItemPacket, with the longer text and ``mark''
%% left blank, since they have no actual item card.
\DECLARESUBTYPE{ItemFold}{ItemPacket}
\PRESETS{ItemFold}{
  \s\MYmark{}
  }


%%%%%
%% ItemEnvelope macros are for items represented by just an envelope.
%% They are a subtype of ItemPacket, with the longer text and ``mark''
%% left blank, since they have no actual item card.
\DECLARESUBTYPE{ItemEnvelope}{ItemPacket}
\PRESETS{ItemEnvelope}{
  \s\MYmark{}
  }


%%%%%
%% ItemLabel macros are for small labels that would get used on
%% physreps, e.g. gun labels.  The ``mark'' is left blank, since
%% it isn't used for these.
\DECLARESUBTYPE{ItemLabel}{Item}
\PRESETS{ItemLabel}{
  \s\MYmark{}
  }


%%%%%
%% \icard[<extras>]{<name>}{<number>}{<text>}
%% \specialicard[<extras>]{<name>}{<number>}{<text>}{<mark>}
%% \itempacket[<extras>]{<name>}{<number>}{<text>}{<mark>}{<contents>}
%% \itemfold{<name>}{<number>}{<text>}{<contents>}
%% \itemenvelope{<name>}{<number>}{<text>}{<contents>}
%% \itemlabel{<name>}{<number>}{<text>}
%%
%% These are wrappers around \INSTANCE, useful for 1-shots.
%%
%% For \icard, \specialicard, and \itempacket, the optional <extras>
%% (in []'s) is for things like \unstash and \bulky{3}.  For example,
%% \icard[\prop\contain{2}]{..}{..}{..}{..} gives an item that has a
%% prop and 3-hands capacity.
%%
%% The last arg (#5) to \specialicard is for anything extra you may
%% want in the ``mark''
\newinstance{Item}{\icard[4][]}{
  \s\MYname{#2}\s\MYnumber{#3}\s\MYtext{#4}#1}
\newinstance{Item}{\specialicard[5][]}{
  \s\MYname{#2}\s\MYnumber{#3}\s\MYtext{#4}\s\MYmark{#5}#1}
\newinstance{ItemPacket}{\itempacket[6][]}{
  \s\MYname{#2}\s\MYnumber{#3}\s\MYtext{#4}\s\MYmark{#5}\s\MYcontents{#6}#1}
\newinstance{ItemFold}{\itemfold[4]}{
  \s\MYname{#1}\s\MYnumber{#2}\s\MYtext{#3}\s\MYcontents{#4}}
\newinstance{ItemEnvelope}{\itemenvelope[4]}{
  \s\MYname{#1}\s\MYnumber{#2}\s\MYtext{#3}\s\MYcontents{#4}}
\newinstance{ItemLabel}{\itemlabel[3]}{
  \s\MYname{#1}\s\MYnumber{#2}\s\MYtext{#3}}


%%%%%%%%%%%%%%%%%%%%%%%%%%%%%%%%%%%%%%%%%%%%%%%%%%%%%%%%%%%%%%%%%%

\NEW{Item}{\iTest}{
  \s\MYname	{Test Item}
  \s\MYnumber	{0000}
  \s\MYtext	{A Test Item Card}
  }

\NEW{ItemPacket}{\iTestPacket}{
  \s\MYname	{Test Item}
  \s\MYnumber	{0000}
  \s\MYtext	{A Test Item with a big red button.  Open packet if
		you press the big red button.}
  \s\MYcontents	{The item beeps at you.}
  }

\NEW{ItemFold}{\iTestFold}{
  \s\MYname	{Test Food}
  \s\MYnumber	{0000}
  \s\MYtext	{open if you eat}
  \s\MYcontents	{It tastes yummy.}
  }

\NEW{ItemEnvelope}{\iTestEnvelope}{
  \s\MYname	{Test Food}
  \s\MYnumber	{0000}
  \s\MYtext	{open if you eat}
  \s\MYcontents	{It tastes yummy.}
  }

\NEW{ItemLabel}{\iTestLabel}{
  \s\MYname	{Test Gun Label}
  \s\MYnumber	{0000}
  \s\MYtext	{Disc gun, loadable to 20 shots.}
  }

\NEW{Item}{\iWhatzit}{
  \s\MYname	{Whatzit}
  \s\MYnumber	{12345}
  \s\MYtext	{If you press it, open packet a.  If you twirl it, open
		packet b.  If you pull it, open packet c.}
  \bulky	{1}
  \s\MYsigns	{\signstrip{a}{it goes ``beep.''}
		\signstrip{b}{it goes ``whoop.''}
		\signstrip{c}{it goes ``bang.''}
		}
  \s\MYabils	{\ability{Stop Crying}{By futzing with the Whatzit, you
		can make babies stop crying.}{I make the baby stop
		crying.}
		}
  }

\DECLARESUBTYPE{NoItem}{Item}
\NEW{NoItem}{\iFireGem}{
  \s\MYname {Fire gem}
}
\NEW{NoItem}{\iWaterGem}{
  \s\MYname {Water gem}
}
\NEW{NoItem}{\iEarthGem}{
  \s\MYname {Earth gem}
}
\NEW{NoItem}{\iAirGem}{
  \s\MYname {Air gem}
}

\def\earth{4}
% Syntax: \butterflyEnvelope{\iItemMacro}{ELEMENT}
% ELEMENT mapping: 1 -> Air, 2 -> Fire, 3 -> Water, 4 -> Earth
\newcommand{\butterflyEnvelope}[2]{
  \NEW{Item}{#1}{
    \def\lefttop{}
    \def\leftbottom{}
    \def\transfer{}
    \s\MYname{\first's Spirit Butterfly}
    \s\MYtext{To summon your spirit butterfly, you must first collect the following elemental gems:\\ \ \\\ifnum #2=1 2 Air gems\else 1 Air gem\fi, \ifnum #2=2 2 Fire gems\else 1 Fire gem\fi, \ifnum #2=3 2 Water gems\else 1 Water gem\fi, \ifnum #2=4 2 Earth gems\else 1 Earth gem\fi\\ \ \\Once you have these gems, you must then reveal to someone a secret about you that they did not already know.\\ \ \\Once you have done so, your spirit butterfly awakens; you may open this envelope. (Your gems' energy is expended and they may no longer be used or traded.)}
  }
}

\butterflyEnvelope{\iMulanButterfly}{4}
\butterflyEnvelope{\iTricksterButterfly}{4}
\butterflyEnvelope{\iHotPersonButterfly}{4}
\butterflyEnvelope{\iGuildmasterButterfly}{3}
\butterflyEnvelope{\iRealPriestButterfly}{3}
\butterflyEnvelope{\iLieutenantButterfly}{3}
\butterflyEnvelope{\iSpyButterfly}{2}
\butterflyEnvelope{\iFakePriestButterfly}{2}
\butterflyEnvelope{\iPatriotButterfly}{2}
\butterflyEnvelope{\iRoyaltyButterfly}{1}
\butterflyEnvelope{\iBastardButterfly}{1}
\butterflyEnvelope{\iServantButterfly}{1}

%%%%%%%%%%%%%%%%%%%%%%%%%%%%%%%%%%%%%%%%%%%%%%%%%%%%%%%%%%%%%%%%%%
